\documentclass[12pt,letterpaper]{article}
%\usepackage{ifpdf,mla}
\usepackage{amssymb} % for $\blacksquare$ and $\therefore$
\usepackage[letterpaper, margin=1in]{geometry} % for margins
\usepackage{setspace} % for \singlespace
\usepackage{enumitem} % for [leftmargin=]
\usepackage{adjustbox}
\usepackage{tabularx} % for proofs spanning multiple pages
\usepackage{longtable}
\usepackage{tabu}
\usepackage{longtable}
\newcommand{\fs}{\textrm{~for some~}}
\newcommand{\lcm}{\textrm{lcm}}
\newenvironment{proof}{
\mbox{}\vspace*{-1.68\baselineskip}
%\setlength{\LTleft}{\leftmargin}
\begin{longtable}[l]{@{} l l}
}{
\tiny {$~\blacksquare$}
\end{longtable}
}
\begin{document}
\singlespacing % remove if no package setspace

%%%%%%%%%%%%%%%%%%%%%%%%%%%%%%%%%%%%%%%%%%%%%%%%%%%%%%%%%%%%%%%%%%%%%%%%%%%%%%%
% Title
%%%%%%%%%%%%%%%%%%%%%%%%%%%%%%%%%%%%%%%%%%%%%%%%%%%%%%%%%%%%%%%%%%%%%%%%%%%%%%%
\begin{center}
{\Large Sam Grayson: Test 1} \\
\today
\end{center}

All new numbers instantiated with `for some' are assumed to be integers.

\begin{enumerate}[leftmargin=0cm]

\item 
$\{a, b \in \mathbb{N} \wedge a|b \wedge b|a\} \rightarrow (a = b) $

\begin{proof}
$a = bc \fs c$ \\
$b = da \fs d$ & Definition of divides \\
$a = cda$ & Substitution \\
$cd = 1$ & Identity property \\
$c = \pm 1 \wedge d = \pm 1$ & Algebra \\
$a = \pm b$ & Substitution \\
$a = b$ & Eliminating extraneous solutions \\
& (noting $a, b \in \mathbb{N}$)
\end{proof}

\item 
$\{a, b, c \in \mathbb{Z} \wedge c > 0 \wedge a \equiv b \pmod c\} \rightarrow (a, c) = (b, c) $

\begin{proof}
$c|(b - a)$ & Definition of modulo \\
$cn = b - a$ & Definition of divides \\
$(a, c) = d$ & Let \\
$(a, c)|a \wedge (a, c)|c$ & Definition of GCD \\
\end{proof}

\item 
$\{a, b, d \in \mathbb{Z} \wedge (a \neq 0 \vee b \neq 0) \wedge d > 0 \wedge d|a \wedge d|b\} \rightarrow d|(a, b)$

\begin{proof}
$d \neg (a, b)$ & Assume for contradiction \\
$(a, b)|a \wedge (a, b)|b$ & Definition of GCD \\
$m (a, b) = a \wedge n (a, b) = b \fs m, n$ & Definition of divides \\
$(m, n) = 1$ & Test question 4 \\
$d|m (a, b) \wedge d|n (a, b)$ & Substitution \\
$(d, (a, b)) = 1$ & Contradictive assumption\\
$d|m \wedge d|n$ & Theorem 1.41 \\
$(m, n) > d$ & Definition of GCD \\
This contradicts $(m, n) = 1$ \\
$a | (a, b)$ & Contradiction
\end{proof}

$\gcd(0, 0)$ is undefined. That is why we must specify that $a$ and $b$ are not both zero.

\item
$d = (a, b) \rightarrow (\frac{a}{d}, \frac{b}{d}) = 1$

\begin{proof}
$d|a \wedge d|b$ & Definition of GCD \\
$d \frac{a}{d} = a \wedge d \frac{b}{d} = b \fs \frac{a}{d}, \frac{b}{d}$ & Definition of divides \\
$c = (\frac{a}{d}, \frac{b}{d}) \wedge \fs c $ & Let \\
$c \neq 1$ & Assume for contradiction \\
$c|\frac{a}{d} \wedge c|\frac{b}{d}$ & Definition of GCD \\
$c \frac{a}{dc} = \frac{a}{d} \wedge c \frac{b}{dc} = \frac{b}{d} \fs \frac{a}{dc}, \frac{b}{dc}$ & Definition of divides \\
$d c \frac{a}{dc} = a \wedge d c \frac{b}{dc} = b$ & Substitution \\
$dc | a \wedge dc | b$ \\
We don't know if dc is positive or negative \\
Therefore I try both \\
$-d c (-\frac{a}{dc}) = a \wedge -d c (-\frac{b}{dc}) = b$ & Substitution \\
$-dc | a \wedge -dc | b$ & Definition of divides \\
$dc > d \vee -dc > d$ & Property of inequality \\
$(a, b) > dc \vee (a, b) > -dc$ & Definition of GCD \\
\parbox[t]{11cm}{Either way, I have found a common divisor (namely $dc$ or $-dc$) greater than $d$. This contradicts the definition of GCD.}\\
$c = 1$ & Contradiction \\
$1 = (\frac{a}{d}, \frac{b}{d})$ & Substitution
\end{proof}

\item Let $a = 6$, $b = 2$, $c = 3$.
\begin{itemize}
\item $a|(bc)$ since $6 | 6$
\item $a \not | b$ since $6 \not | 2$
\item $a \not | c$ since $6 \not | 3$
\end{itemize}

\item
$\{a, b, c, n_1, n_2 \in \mathbb{Z} \wedge a \equiv b \pmod{n_1} \wedge a \equiv c \pmod{n_2}\} \rightarrow b \equiv c \pmod{(n_1, n_2)}$

\item % Question 7
\begin{enumerate} % part  a, b, c
\item 
\begin{enumerate} % steps 1, 2, 3...
\item $2072 = 1813 \cdot 1 + 259$. Therefore $(2072, 1813) = (1813, 259)$
\item $1813 = 259 \cdot 7 + 0$. Therefore $(1813, 259) = (259, 0) = 259$
\item Therefore $(2072, 1813) = 259$
\end{enumerate}
\item 
$2072 = 1813 \cdot 1 + 259$ \\
$1813 = 259 \cdot 7$ \\
$2072 = (259 \cdot 7) + 259$ \\
$2072 = 259 \cdot 8$ \\
$2072 x + 1813 y = 2048$ \\
$259 \cdot 8 x + 259 \cdot 7 y = 2048 = 11 \cdot 259$ \\
$259 \cdot (8x + 7y) = 259 \cdot 11$ \\
$259 \cdot 11 \cdot (8 + (-7)) = 259 \cdot 11$ \\
$259 \cdot (8 \cdot 11 + 7 \cdot (-11)) = 2849$ \\
$8 \cdot 259 \cdot 11 + 7 \cdot 259 \cdot (-11) = 2849$ \\
$2072 \cdot 11 +  1813 \cdot (-11) = 2849$ \\
$x = 11 \wedge y = -11$ \\
\item 
$259 \cdot (8x + 7y) = 259 \cdot 11$ \\
$259 \cdot 11 \cdot (8 + (-7) + 0) = 259 \cdot 11$ \\
$259 \cdot 11 \cdot (8 \cdot 1 + 7 \cdot (-1) + 0) = 259 \cdot 11$ \\
$259 \cdot 11 \cdot (8 \cdot 1 + 7 \cdot (-1) + 8 \cdot 7 - 7\cdot 8) = 259 \cdot 11$ \\
$259 \cdot 11 \cdot (8 \cdot (1 - 7) + 7 \cdot (-1 + 8)) = 259 \cdot 11$ \\
$259 \cdot 11 \cdot (8 \cdot (-6) + 7 \cdot 7) = 2849$ \\
$259 \cdot (8 \cdot (-66) + 7 \cdot 77) = 2849$ \\
$8 \cdot 259 \cdot (-66) + 7 \cdot 259 \cdot 77 = 2849$ \\
$2072 \cdot (-66) + 1813 \cdot 77 = 2849$ \\
$x = -66 \wedge y = 77$ \\
\end{enumerate}

\item
$\{a, b, c \in \mathbb{Z} \wedge (a \neq 0 \vee b \neq 0) \wedge c \neq 0\} \rightarrow (ca, cb) = |c|(a, b)$

$|x| = \left \{ \begin{array}{cc} x & x \geq 0 \\ -x & x < 0\end{array} \right. $

\begin{proof}
$c \in \mathbb{Z} \backslash \{0\}$ & Premise \\
Either $c>0 \vee c = 0 \vee c < 0$ & Trichotomy \\
$c \neq 0$ & Premise \\
Temporarily assume $c > 0$ \\
$|c| = c$ & Definition of absolute value \\
$|c| \in \mathbb{N}$ & Definition of $\mathbb{N}$ \\
$(ca, cb) = (|c|a, |c|b)$ & Substitution \\
$(|c|a, |c|b) = |c|(a, b)$ & Theorem 1.55 \\
Temporarily assume $c < 0$ \\
$|c| = -c$ & Definition of absolute value \\
$-c > 0$ & Property of inequalities \\
& (multiplying by $-1$ flips direction) \\
$-c \in \mathbb{Z}$ & Definition of $\mathbb{N}$ \\
$(-ca, -cb) = -c(a, b)$ & Theorem 1.55 \\
$(|c|a, |c|b) = |c|(a, b)$ & Substitution \\
All possibilities were tried \\
$(|c|a, |c|b) = |c|(a, b)$ & Constructive Dilemma
\end{proof}

\item 

\item {\textbf{Problem:} $\forall n \in \mathbb{N} \{ 6|(n^3 + 5n)\}$

\textbf{Lemma:} $6|(3n(n^2 + 1))$

Assume $n$ is odd, such that $n = 2k+1$. $3n(n^2 + 1) = 3n(4k^2 + 2k + 1 + 1) = 6n(2k^2 + k + 1)$. Therefore $3n(n^2 + 1)$ is divisible by $6$.

Assume $n$ is even, such that $n = 2k$. $3n(n^2 + 1) = 6k(n^2 + 1)$. Therefore $3n(n^2 + 1)$ is divisible by $6$.

Therefore, for any integer $n$, $3n(n^2 + 1)$ is divisible by $6$. $\square$

\textbf{Proof:}

Let $n = 1$. $n^3 + 5n$ is divisible by $6$, because $n^3 + 5n = 6$.

Assume $n^3 + 5n$ is divisible by $6$.

$(n+1)^3 + 5(n+1) = (n^3 + 3n^2 + 3n + 1) + (5n + 5) = (n^3 + 5n) + (3n^2 + 3n + 6)$

Therefore $(n+1)^3 + 5(n+1)$ is the sum of things divisible by six (namely $(n^3 + 5n)$, $3n(n^2 + 1)$, and $6$).

Therefore, by the induction axiom, $n^3 + 5n$ is divisible by $6$.
}

\end{enumerate}

\end{document}

%%% Local Variables:
%%% mode: latex
%%% TeX-master: t
%%% End:
