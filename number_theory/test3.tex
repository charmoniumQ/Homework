\documentclass[12pt,letterpaper]{article}
\usepackage[utf8]{inputenc}
%\usepackage{ifpdf,mla}
%\usepackage{showframe}
\usepackage[english]{babel}
\usepackage{amsmath}
\usepackage{amssymb}
\usepackage[letterpaper, margin=1in]{geometry}
\usepackage{setspace}
\usepackage{enumitem}
\usepackage[pdftex]{graphicx}
\usepackage{longtable}
\usepackage{fancyhdr}
\usepackage{lastpage}
\usepackage{cancel}
\usepackage{minted}
\usepackage{python}

\pagestyle{fancy}
\fancyhead{}
\fancyfoot{}
\fancyhead[HR]{\thepage\ of \pageref{LastPage}}

\newcommand{\fs}{\textrm{\ for some\ }}
\DeclareMathOperator{\lcm}{lcm}
\DeclareMathOperator{\pf}{pf}
\newenvironment{proof}{
\textbf{Proof:} \\
\mbox{}\vspace*{-1.68\baselineskip}
\setlength\LTleft{\leftmargin+20pt}
\setlength\LTright\fill
\begin{longtable}{@{} ll}
}{
\tiny {$~\blacksquare$}
\end{longtable}
}

\begin{document}

% Definition of Hilbert number
% 5.a: explain process

\doublespacing
\begin{center}
{\Large Test 2} \\[14pt]
{\large Sam Grayson} \\[0pt]
{\today} \\
\end{center}

\singlespacing
\setlength{\parindent}{0pt}

\begin{enumerate}[leftmargin=0mm]
\item
\begin{tabular}[t]{l l}
$0 \equiv 3 \cdot 0 \pmod{17}$ \\
$1 \equiv 3 \cdot 6 \pmod{17}$ \\
$2 \equiv 3 \cdot 12 \pmod{17}$ \\
$3 \equiv 3 \cdot 18 \pmod{17}$ \\
$4 \equiv 3 \cdot 24 \pmod{17}$ \\
$5 \equiv 3 \cdot 30 \pmod{17}$ \\
$6 \equiv 3 \cdot 36 \pmod{17}$ \\
$7 \equiv 3 \cdot 42 \pmod{17}$ \\
$8 \equiv 3 \cdot 48 \pmod{17}$ \\
$9 \equiv 3 \cdot 54 \pmod{17}$ \\
$10 \equiv 3 \cdot 60 \pmod{17}$ \\
$11 \equiv 3 \cdot 66 \pmod{17}$ \\
$12 \equiv 3 \cdot 72 \pmod{17}$ \\
$13 \equiv 3 \cdot 78 \pmod{17}$ \\
$14 \equiv 3 \cdot 84 \pmod{17}$ \\
$15 \equiv 3 \cdot 90 \pmod{17}$ \\
$16 \equiv 3 \cdot 96 \pmod{17}$ \\
\end{tabular}

\(\{0, 18, 36, 54, 72, 90, 108, 126, 144, 162, 180, 198, 216, 234, 252, 270, 288\}\) forms a complete residue system mod \(17\). I generated the table above using the following Python code. For an explanation of Python code in general and the source for \mintinline[linenos,mathescape,baselinestretch=0.9,fontsize=\small]{python}{linear_diophantine()}, please read 3.23 in my notebook.

\begin{minted}[linenos,mathescape,baselinestretch=0.9,fontsize=\small]{python}
from tools import linear_diophantine
CRS = []
for n in range(17):
    (x_0, y_0), (r_x, r_y) = linear_diophantine(3, -17, n)
    # now we have $3 x_0 - 17 y_0 = n$
    # output $n \equiv 3 \cdot x_0 \pmod{17}$
    print (r'${n} \equiv 3 \cdot {x_0} \pmod{{17}}$ \\'.format(**locals()))
    CRS.append(3 * x_0)

# output the whole CRS, seperated by commas
print (', '.join(map(str, CRS)))
\end{minted}

\item Find \(2^{100} \pmod 9\)

All congruence statements are taken mod \(9\).

\begin{tabular}[t]{l l}
\(2^{100}\) & \(\equiv \quad ?\)\\
& \(\equiv 2^{3 \cdot 33 + 1}\) \\
& \(\equiv (2^3)^{33} \cdot 2^1\) \\
& \(\equiv 8^{33} \cdot 2 \) \\
& \(\equiv (-1)^{33} \cdot 2\) \\
& \(\equiv -1 \cdot 2\) \\
& \(\equiv 7\) \\
\end{tabular}

\item \textbf{Theorem:} For any CRS mod \(m\), multiplying by a natural coefficient coprime to \(m\) produces a new CRS.

Let \(\{r_1, r_2, \dots, r_m\}\) form a CRS mod \(m\). Let \(a \in\ mathbb N\) where \(\gcd(a, m) = 1\).

\textbf{Proof:} Theorem 3.17 states that \(m\) unique (unique mod \(m\)) elements describe a CRS. Begin with \(\{r_1, r_2, \dots, r_m\}\) which is a CRS, so the elements are unique by definition. \(r_i \not\equiv r_j \rightarrow a r_i \not\equiv a r_j\) by the inverse of theorem 


\end{enumerate}

\end{document}

%%% Local Variables:
%%% mode: latex
%%% TeX-master: t
%%% End:
